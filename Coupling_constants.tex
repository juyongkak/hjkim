\documentclass[aps,prd,preprintnumbers,showpacs,showkeys,nofootinbib,
superscriptaddress,fleqn,floatfix,tightenlines, 10pt]{revtex4-1}
\usepackage{amsmath,amsfonts,amssymb,amscd,amsxtra,amsthm}
\usepackage{graphicx}  % Include figure files
\usepackage{epstopdf}
\usepackage{dcolumn}  % Align table columns on decimal point
\usepackage{bm}          % bold math
\usepackage{slashed}
	\usepackage[utf8]{inputenc}
%raggedright - Each line of the caption will be moved to
%the left margine
%%%%%%%%%%%%%%%%%%%%%%%%%%%%%%%%%%%
\usepackage{booktabs}
\usepackage[normalem]{ulem} % \sout{old text} for strikeout
\usepackage[dvipsnames]{xcolor} % For blue in-text comments and
                                % additions
\usepackage{enumitem}
\usepackage{array}
\usepackage{slashed}
\usepackage{tikz}
\usepackage{float}
\usepackage{multirow}
%\usepackage{subfig}
\renewcommand\sout{\bgroup \color{red} \ULdepth=-.5ex \ULset}
\newcommand{\com}[1]{{\sf\color[rgb]{0,0,1}{#1}}}
%%%%%%%%%%%%%%%%%%%%%%%%%%%%%%%%%%%
%-------------------------------------------------
\begin{document}
\preprint{INHA-NTG-mm/yyyy}
\title{$DD\sigma$ and $D^* D^* \sigma$ coupling constants}
%-------------------------------------------------
\author{Hee-Jin Kim}
\affiliation{Department of Physics, Inha University, Incheon 402-751,
Korea}
\date{\today}
\maketitle

% ===================================================================================================================

\section{Effective Lagrangian}
 The effective Lagrangian we use here is as follows :
\begin{align} \label{Lagrangian}
	\mathcal{L}_{D D^* \pi} = g_{D D^* \pi} D^*_\mu \bm{\tau} \cdot \partial^\mu \bm{\pi} D
	+ g_{D D^* \pi} D \bm{\tau} \cdot \partial^\mu \bm{\pi} D^*_\mu
\end{align}
\begin{align} \label{Lagrangian_D*}
	\mathcal{L}_{D^* D^* \pi} = i g_{D^* D^* \pi} \varepsilon^{\alpha\beta\mu\nu} D^*_\alpha \bm{\tau} \cdot \partial_\mu \bm{\pi}
	\partial _\nu D^*_\beta,
\end{align}
where $\bm{\tau}$ is pauli matrices for isospin. One can proof the hermitivity of $\mathcal{L}_{D^* D^* \pi}$ as
\begin{align}\label{hermitivity_D*}
	\left[i \varepsilon^{\alpha \beta \mu \nu} D_{\alpha}^\ast \bm{\tau} \cdot \partial_\mu \bm{\pi} \partial_\nu D_{\beta}^\ast \right]^\dagger
	= -i \varepsilon^{\alpha \beta \mu \nu} \partial_\nu D_{\beta}^\ast \bm{\tau} \cdot \partial_\mu \bm{\pi} D_{\alpha}^\ast
	= i \varepsilon^{\alpha \beta \mu \nu} D_{\alpha}^\ast \bm{\tau} \cdot \partial_\mu \bm{\pi} \partial_\nu D_{\beta}^\ast.
\end{align}

\section{The $\pi D \rightarrow \pi D$ amplitudes}
% ===================================================================================================================
\subsection{Kinematics}
Let $p_1$ and $k_1$ be four-momenta for the initial $D$ meson and $\pi$ and $p_2$ and $q_2$ be those of final particles,
respectively. We describe the kinematics in the center of mass frame for the calculation.
\begin{align} \label{ddpipi_momenta}
	  & p_1 = (E, \bm{p}), \quad k_1 = (\omega, -\bm{p}) \cr
	  & p_2 = (E, \bm{k}), \quad k_2 = (\omega, -\bm{k})
\end{align}
We define the kinematic variables as below.
\begin{align}
	  s &= (p_1 + k_1)^2 = m_D^2 + m_\pi^2 + 2 p_1 \cdot k_1 = (E + \omega)^2	\cr
	  t &= (p_1 - p_2)^2 = 2 m_D^2 - 2 p_1 \cdot p_2  = -2|\bm{p}|^2 (1-\cos{\theta})	\cr
	  u &= (p_1 - k_2)^2 = 2m_D^2 + 2m_\pi^2 - s - t = m_D^2 + m_\pi^2 - 2p_1 \cdot k_2
\end{align}
and
\begin{align}
	p_1 \cdot k_1 &= p_2 \cdot k_2 = (s - m_D^2 - m_\pi^2)/2 \cr
	p_1 \cdot p_2 &= k_1 \cdot k_2 = m_D^2 - t/2 \cr
	p_1 \cdot k_2 &= p_2 \cdot k_1 = -(u - m_D^2 - m_\pi^2)/2.
\end{align}

\subsubsection{s-channel Feynman diagram}
\begin{figure}[H] \label{dpidpi_s_diagram}
	\centering
	\includegraphics[scale=0.5]{coup_born_s-crop.pdf}
	\caption{$s$-channel Feynman diagram}
	\label{fig1}
\end{figure}
%=================================
% the vertex feynman rule should be derived here!
%=================================
The $s$-channel Feynman diagram is depicted as Fig.(\ref{dpidpi_s_diagram}) and
corresponding Feynman rules are derived as Fig.(ref). We can thus compute the Feynman amplitudes :
\begin{align}
	i \mathcal{M}_{ij}^s = (g_{DD^* \pi} k_1^\mu) \tau_i \times
	\frac{-i \left(g_{\mu\nu} - \frac{q_\mu q_\nu}{m_{D^*}^2}\right)}{q^2-m_{D^*}^2} \times (-g_{DD^* \pi} k_2^\nu) \tau_j
	= i g_{DD^* \pi}^2 \frac{k_1 \cdot k_2 - \frac{(k_1 \cdot q)(k_2 \cdot q)}{m_{D^*}^2}}{s-m_{D^*}^2} \tau_i \tau_j .
\end{align}
From Eq.(\ref{ddpipi_momenta}) we have
\begin{align}
	k_1 \cdot k_2 = m_\pi^2 + |\bm{p}|^2 (1-\cos{\theta}) = m_\pi^2 - \frac{t}{2}
\end{align}
and
\begin{align}
	k_1 \cdot q &= k_1 \cdot (k_1 + p_1) = k_1^2 + p_1 \cdot k_1 = m_\pi^2 + (s - m_D^2 - m_\pi^2)/2 = (s - m_D^2 + m_\pi^2)/2 \cr
	k_2 \cdot q &= k_2 \cdot (k_2 + p_2) = k_2^2 + p_2 \cdot k_2 = (s - m_D^2 + m_\pi^2)/2
\end{align}
So that, the $s$-channel amplitude is
\begin{align}\label{ddpipi_s}
	i \mathcal{M}_{\alpha\beta}^s = i g_{DD^* \pi}^2
	\frac{m_\pi^2 - t/2 - (s - m_D^2 + m_\pi^2)^2/4 m_{D^*}^2}{s - M_{D^*}^2} \tau_i \tau_j.
\end{align}

% ===================================================================================================================
\subsubsection{u-channel diagram}
\begin{figure}[H]
	\centering
	\includegraphics[scale=0.5]{coup_born_u-crop.pdf}
	\caption{$u$-channel Feynman diagram}
	\label{fig2}
\end{figure}
As we did for the $s$-channel, the $u$-channel amplitude may be obtaind as
\begin{align}
	i\mathcal{M}_{ij}^u &= (-g_{DD^* \pi} k_2^\mu) \tau_j \times
	\frac{-i\left(g_{\mu\nu} - \frac{q_\mu q_\nu}{m_{D^*}^2}\right)}{q^2-m_{D^*}^2} \times
	g_{DD^* \pi} k_1^\nu \tau_i = i g_{DD^* \pi}^2 \frac{k_1 \cdot k_2 -
	\frac{(k_1 \cdot q)(k_2 \cdot q)}{m_{D^*}^2}}{u-m_{D^*}^2} \tau_j \tau_i
\end{align}
Since the four-momenta are conserved at every vertices, we have
\begin{align}
	k_1 \cdot q &= k_1 \cdot (p_2 - k_1) = p_2 \cdot k_1 - m_\pi^2 = - (u - m_D^2 - m_\pi^2)/2 - m_\pi^2 = - (u - m_D^2 + m_\pi^2)/2 \cr
	k_2 \cdot q &= k_2 \cdot (p_1 - k_2) = p_1 \cdot k_2 - m_\pi^2 = - (u - m_D^2 + m_\pi^2)/2.
\end{align}
So that we may write the $u$-channel Feynman amplitude :
\begin{align}\label{dpidpi_u}
	i g_{DD^* \pi}^2 \frac{m_\pi^2 - t/2 - (u - m_D^2 - m_\pi^2)^2/4m_{D^*}^2}{u-m_{D^*}^2} \tau_j \tau_i.
\end{align}
Thus the amplitudes for $\pi D \rightarrow \pi D$ process are
\begin{align}
	i \mathcal{M}_{\alpha\beta}^s &= i g_{DD^* \pi}^2
	\frac{m_\pi^2 - t/2 - (s - m_D^2 + m_\pi^2)^2/4 m_{D^*}^2}{s - M_{D^*}^2} \tau_i \tau_j \cr
	i\mathcal{M}^u_{\alpha\beta} &= i g_{DD^* \pi}^2
	\frac{m_\pi^2 - t/2 - (u - m_D^2 - m_\pi^2)^2/4m_{D^*}^2}{u-m_{D^*}^2} \tau_j \tau_i.
\end{align}

\section{The $D \bar{D} \rightarrow \pi\pi$ amplitude}

\begin{figure}[H]
	\centering
	\includegraphics[scale=0.8]{coup_born_tu-crop.pdf}
	\caption{$D \bar{D} \rightarrow \pi\pi$ Born diagrams}
	\label{fig3}
\end{figure}
% ===================================================================================================================
\subsection{Kinematics}
Let the momenta of $D$ and $\bar{D}$ to be $p_1$ and $\bar{p}_2$ and the momenta of off-shell pions
to be $\bar{k}_1$ and $k_2$, respectively.
\begin{align}
	  p_1 &= (E_p, \bm{p}), \quad \bar{p}_2 = (E_p, -\bm{p}) \cr
	  \bar{k}_1 &= (\omega_1, \bm{k}), \quad k_2 = (\omega_2, -\bm{k})
\end{align}
We define the Mandelstam variables again. Note that basically the $s$ used to be square of the center
of mass energy, however we used the $s$ as the momentum transfer here for our purpose.
\begin{align}\label{mandelstam_ddpipi}
	s &= (p_1 - \bar{k}_1)^2 = p_1^2 + \bar{k}_1^2 - 2 p_1 \cdot \bar{k}_1 = m_D^2 + \omega_1^2 - |\mathbf{k}|^2 - 2 p_1 \cdot k_1 \cr
	t &= (p_1 + \bar{p}_2)^2 = p_1^2 + \bar{p}_2^2 + 2 p_1 \cdot \bar{p}_2 = 2 m_D^2 + 2p_1 \cdot \bar{p}_2 = 4E_p^2\cr
	&= (\bar{k}_1 + k_2)^2 = \bar{k}_1^2 + k_2^2 + 2 \bar{k}_1 \cdot k_2
	= \omega_1^2 + \omega_2^2 - 2 |\mathbf{k}|^2 + 2 \bar{k}_1 \cdot k_2  = (\omega_1 + \omega_2)^2 \equiv 4 \bar{\omega}^2\cr
	% t = 4E_p^2 = 4\bar{\omega}^2 is not correct. The four-momenta are not conserved!!
	u &= (p_1 - k_2)^2 = p_1^2 + k_2^2 - 2 p_1 \cdot k_2 = m_D^2 + \omega_2^2 - |\mathbf{k}|^2 - 2 p_1 \cdot k_2
\end{align}
The off-shell pion momenta :
\begin{align}
	\bar{k}_1^2 = k_2^2 = \frac{t}{4} - |\mathbf{k}|^2
\end{align}
%=====================================================
It is convenient to write down the momentum constractions in terms of the Mandelstam variables
for our calculations.
% Off shell pion case
%\begin{align}
%	p_1 \cdot \bar{p}_2 &= m_D^2 - t/2	\\
%	\bar{k}_1 \cdot k_2 &= \omega_k^2 + |\bm{k}|^2	(???)\\
%	p_1 \cdot \bar{k}_1 &=
%\end{align}
%
% On shell pion case
\begin{align}
	p_1 \cdot \bar{p}_2 &= m_D^2 - t/2 \cr
	\bar{k}_1 \cdot k_2 &= \frac{t}{4} + |\mathbf{k}|^2 \cr
	p_1 \cdot \bar{k}_1 &= (m_D^2 + \omega^2 - |\mathbf{k}|^2 - s)/2 \cr
	p_1 \cdot k_2 &= (m_D^2 + \omega^2 - |\mathbf{k}|^2 - u)/2
\end{align}
The $s$-channel amplitude :
\begin{align}
	i\mathcal{M}_{ij}^s = g_{DD^* \pi} \bar{k}_1^\mu \tau_i \times \frac{-i\left(g_{\mu\nu} -
	\frac{q_\mu q_\nu}{m_{D^*}^2} \right)}{q^2 - m_{D^*}^2} \times g_{DD^* \pi} k_2^\nu \tau_j
	= -i g_{DD^* \pi}^2 \frac{\bar{k}_1 \cdot k_2 - \frac{(\bar{k}_1 \cdot q)(k_2 \cdot q)}
	{m_{D^*}^2}}{s - m_{D^*}^2} \tau_i \tau_j
\end{align}
%
We will drop the symbol of absolute value for breivity.\\\\
1) $g_{\mu\nu}$ term
\begin{align}
	\bar{k}_1 \cdot k_2 = \frac{t}{4} + \mathbf{k}^2
\end{align}
%
2) $q_\mu q_\nu$ term
% Off shell pion case
%\begin{align}
%	\bar{k}_1 \cdot q &= \bar{k}_1 \cdot (p_1 - \bar{k}_1) = p_1 \cdot \bar{k}_1 - \bar{k}_1^2
%	= (m_D^2 + \omega_1^2 - |\mathbf{k}|^2 - s)/2 - \omega_1^2 + |\mathbf{k}|^2
%	= (m_D^2 - \omega_1^2 + |\mathbf{k}|^2 - s)/2 \cr
%	k_2 \cdot q &= k_2 \cdot (\bar{p}_2 + k_2) = \bar{p}_2 \cdot k_2 + k_2^2
%	= (m_D^2 - \omega_2^2 + |\mathbf{k}|^2 - s)/2
%\end{align}
% On shell pion case
\begin{align}
	\bar{k}_1 \cdot q &=  \bar{k}_1 \cdot (p_1 - \bar{k}_1) = p_1 \cdot \bar{k}_1 - \bar{k}_1^2 \cr
	&= (m_D^2 + \omega^2 - \mathbf{k}^2 - s)/2 - (\omega^2 - \mathbf{k}^2)
	= (m_D^2 - \omega^2 + \mathbf{k}^2 - s)/2 = (4m_D^2 - t - 4s + 4\mathbf{k}^2)/8 \cr
	k_2 \cdot q &= k_2 \cdot (k_2 - \bar{p}_2) = k_2^2 - \bar{p}_2 \cdot k_2 \cr
	&= (\omega^2 - \mathbf{k}^2 - (m_D^2 + \omega^2 - \mathbf{k}^2 - s)/2
	= -(m_D^2 - \omega^2 + \mathbf{k}^2 - s)/2 = -(4m_D^2 - t - 4s + 4\mathbf{k}^2)/8
\end{align}
with $q = \bar{k}_1 - p_1 = \bar{p}_2 - k_2$. Therefore
% Off shell pion case
%\begin{align}
%	i\mathcal{M}^s_{ij} = i g_{DD^* \pi}^2 \frac{\bar{k}_1 \cdot k_2 -
%	(m_D^2 - \omega_1^2 + |\mathbf{k}|^2 - s)(m_D^2 - \omega_2^2 + |\mathbf{k}|^2 - s)/4M_{D^*}^2}{s - M_{D^*}^2} \tau_i \tau_j
%\end{align} with $q = \bar{k}_1 - p_1 = \bar{p}_2 - k_2$.
\begin{align}
	i\mathcal{M}^s_{ij} = -i g_{DD^* \pi}^2 \frac{t/4 + \mathbf{k}^2 +
	(4m_D^2 - t - 4s + 4\mathbf{k}^2)^2/64M_{D^*}^2}{s - M_{D^*}^2} \tau_i \tau_j
\end{align}
In the similar way, we can also calculate the $u$-channel diagram.
\begin{align}
	i\mathcal{M}_{ij}^u &= g_{DD^* \pi} k_2^\mu \tau_j \times \frac{-i \left(g_{\mu\nu}
	 - \frac{q_\mu q_\nu}{m_{D^*}^2}\right)}{q^2 - m_{D^*}^2} \times g_{DD^* \pi} \bar{k}_1^\nu \tau_i \cr
	 &= -i g_{DD^* \pi}^2 \frac{\bar{k}_1 \cdot k_2 - \frac{(\bar{k}_1 \cdot q)(k_2 \cdot q)}{m_{D^*}^2}}{u-m_{D^*}^2} \tau_j \tau_i
\end{align}
The only difference is that $q$ is changed to $p_1 - k_2 = \bar{k}_1 - \bar{p}_2$.
\begin{align}
	\bar{k}_1 \cdot q &= \bar{k}_1 \cdot (\bar{k}_1 - \bar{p}_2) = \bar{k}_1^2 - \bar{p}_2 \cdot \bar{k}_1 \cr
	&= \omega^2 - \mathbf{k}^2 - (m_D^2 + \omega^2 - \mathbf{k}^2 - u)/2
	= - (m_D^2 - \omega^2 + \mathbf{k}^2 - u)/2 = -(4m_D^2 - t - 4u - 4\mathbf{k}^2)/8 \cr
	k_2 \cdot q &= k_2 \cdot (p_1 - k_2) = p_1 \cdot k_2 - k_2^2 \cr
	&= (m_D^2 - \omega^2 - \mathbf{k}^2 - u)/2 - (\omega^2 - \mathbf{k}^2)
	= (m_D^2 - \omega^2 - \mathbf{k}^2 -u)/2 = (4m_D^2 - t - 4u - 4\mathbf{k}^2)/8
\end{align}
The $u$-channel amplitude :
\begin{align}
	i\mathcal{M}_{ij}^u = -ig_{DD^* \pi}^2 \frac{t/4 - \mathbf{k}^2 +
	(4m_D^2 - t - u - 4\mathbf{k}^2)^2/64m_{D^*}^2}{u-m_{D^*}^2}\tau_j \tau_i
\end{align}
%
So that
\begin{align}
	i\mathcal{M}^s_{ij} &= -i g_{DD^* \pi}^2 \frac{t/4 + \mathbf{k}^2 +
	(4m_D^2 - t - 4s + 4\mathbf{k}^2)^2/64M_{D^*}^2}{s - M_{D^*}^2} \tau_i \tau_j\\
	i\mathcal{M}_{ij}^u &= -ig_{DD^* \pi}^2 \frac{t/4 + \mathbf{k}^2 +
	(4m_D^2 - t - 4u - 4\mathbf{k}^2)^2/64m_{D^*}^2}{u-m_{D^*}^2}\tau_j \tau_i .
\end{align}
We thus obtain the total amplitude.
\begin{align}
	\mathcal{M}_{ij}^{\mathrm{tot}} = -g_{DD^* \pi}^2
	&\left\{ \frac{t/4 + \mathbf{k}^2 + (4m_D^2 - t - 4s + 4\mathbf{k}^2)^2/64M_{D^*}^2}{s - M_{D^*}^2} \tau_i \tau_j \right. \cr
	&\quad \left. + \frac{t/4 + \mathbf{k}^2 + (4m_D^2 - t - 4u - 4\mathbf{k}^2)^2/64m_{D^*}^2}{u-m_{D^*}^2} \tau_j \tau_i \right\}
\end{align}
The amplitudes can be decomposed into two parts by the identity
\begin{align}
	\tau_i \tau_j = \frac{1}{2} \left( \{\tau_i, \tau_j\} + [\tau_i, \tau_j] \right) = \delta_{ij} + \frac{1}{2}[\tau_i, \tau_j].
\end{align}
Hence, we may write
\begin{align}
	\mathcal{M}_{ij}^{\mathrm{tot}} = \mathcal{M}^{(+)} \delta_{ij} + \mathcal{M}^{(-)} \frac{1}{2} [\tau_i, \tau_j],
\end{align}
where
\begin{align} \label{dpidpi_pmamp}
	\mathcal{M}^{(+)} &\equiv - g_{DD^* \pi}^2 \left\{ \frac{t/4 + \mathbf{k}^2 +
	(4m_D^2 - t - 4s + 4\mathbf{k}^2)^2/64M_{D^*}^2}{s - M_{D^*}^2}
	+ \frac{t/4 + \mathbf{k}^2 + (4m_D^2 - t - 4u - 4\mathbf{k}^2)^2/64m_{D^*}^2}{u-m_{D^*}^2} \right\}\cr
	\mathcal{M}^{(-)} &\equiv - g_{DD^* \pi}^2 \left\{ \frac{t/4 + \mathbf{k}^2 -
	(4m_D^2 - t - 4s + 4\mathbf{k}^2)^2/64M_{D^*}^2}{s - M_{D^*}^2}
	- \frac{t/4 + \mathbf{k}^2 + (4m_D^2 - t - 4u - 4\mathbf{k}^2)^2/64m_{D^*}^2}{u-m_{D^*}^2} \right\}.
\end{align}
%\begin{align}
%	  & i\mathcal{M}^s_{\alpha\beta} = i \left( \frac{2g}{f_\pi} \right)^2 \frac{
%	m_\pi^2 - (t/2) - (s - M_D^2 + m_\pi^2)^2/4M_{D^*}^2
%	}{s - M_{D^*}^2} \tau_i \tau_j \\
%	  & i\mathcal{M}^u_{\alpha\beta} = i \left(\frac{2g}{f_\pi} \right)^2 \frac{
%	m_\pi^2 - (t/2) - (u - M_D^2 + m_\pi^2)^2/4M_{D^*}^2
%	}{u - M_{D^*}^2} \tau_j \tau_i
%\end{align}
%The total amplitude :
%\begin{align} \label{D-amplitude}
%	i\mathcal{M}_{\alpha\beta} & = i \left( \frac{2g}{f_\pi} \right)^2 \left[
%	\frac{
%	m_\pi^2 - (t/2) - (s - M_D^2 + m_\pi^2)^2/4M_{D^*}^2
%	}{s - M_{D^*}^2} \tau_i \tau_j
%	\right. \cr
%	& \qquad\qquad\qquad\qquad\qquad\qquad\qquad \left. +\ \frac{
%	m_\pi^2 - (t/2) - (u - M_D^2 + m_\pi^2)^2/4M_{D^*}^2
%	}{u - M_{D^*}^2} \tau_j \tau_i
%	\right]  \cr
%	& = i\mathcal{M}^{(+)}\delta_{\alpha\beta}
%	+ i\mathcal{M}^{(-)}\frac{1}{2} [\tau_i,\tau_j],
%\end{align}
%where
%\begin{align}
%	\mathcal{M}^{(+)} & = \left( \frac{2g}{f_\pi} \right)^2 \left[
%	\frac{
%	m_\pi^2 - (t/2) - (s - M_D^2 + m_\pi^2)^2/4M_{D^*}^2
%	}{s - M_{D^*}^2} + \frac{
%	m_\pi^2 - (t/2) - (u - M_D^2 + m_\pi^2)^2/4M_{D^*}^2
%	}{u - M_{D^*}^2}
%	\right]  \\
%	\mathcal{M}^{(-)} & = \left( \frac{2g}{f_\pi} \right)^2 \left[
%	\frac{
%	m_\pi^2 - (t/2) - (s - M_D^2 + m_\pi^2)^2/4M_{D^*}^2
%	}{s - M_{D^*}^2} - \frac{
%	m_\pi^2 - (t/2) - (u - M_D^2 + m_\pi^2)^2/4M_{D^*}^2
%	}{u - M_{D^*}^2} \right]
%\end{align}
%and
%\begin{align}
%	  & \mathcal{M}_{3/2} = \mathcal{M}^{(+)} - \mathcal{M}^{(-)}  \\
%	  & \mathcal{M}_{1/2} = \mathcal{M}^{(+)} + 2\mathcal{M}^{(-)}.
%\end{align}
%
%$m_\pi =$ 135\,MeV, $M_D =$ 1865\,MeV and $M_{D^*} =$ 2007\,MeV.
%\begin{align}
%  \frac{1}{4M_D^2} \frac{(s - M_D^2 + m_\pi^2)^2}{s - M_{D^*}^2} &=
%  \frac{1}{4M_D^2} \left[
%    \frac{(s - M_D^2)^2}{s-M_{D^*}^2} + 2m_\pi^2 \frac{s-M_D^2}{s-M_{D^*}^2} + \frac{m_\pi^4}{s-M_{D^*}^2}
%  \right]
%\end{align}
%
% ===================================================================================================================
% ===================================================================================================================

\section{The $D^* \bar{D}^* \rightarrow \pi\pi$ amplitude}

\subsection{$D$-exchange}
In this section, we will calculate the $D^* \bar{D}^*$ scattering processes
which exchage $D$- and $D^*$ mesons. First up, we want to start
from the $D$-exchange diagram. These processes can be expressed diagramatically as
\begin{figure}[H]
	\centering
	\includegraphics[scale=0.6]{DstarDstar_t-crop.pdf}
	\caption{$D^* \bar{D}^* \rightarrow \pi\pi$ Born diagrams ($D$-exchange)}
	\label{fig4}
\end{figure}
The $s$- channel :
\begin{align}
	i\mathcal{M}_{ij}^{D,s} & = g_{DD^* \pi} k_1^\mu \epsilon_\mu^{(\lambda)}(p_1) \tau_i
	\times \frac{i}{q^2 - m_D^2} \times g_{DD^* \pi} k_2^\nu \epsilon_\nu^{(\lambda')}(p_2) \tau_j \cr
	& = i g_{DD^* \pi}^2 \epsilon_\mu^{(\lambda)} (p_1) \epsilon_\nu^{(\lambda')} (p_2)
	\frac{k_1^\mu k_2^\nu}{s - M_D^2} \tau_i \tau_j
\end{align}
The $u$-channel :
\begin{align}
	i\mathcal{M}_{ij}^{D,u} & = g_{DD^* \pi} k_2^\mu \epsilon_\mu^{(\lambda)}(p_1) \tau_j
	\times \frac{i}{q^2 - m_D^2} \times g_{DD^* \pi} k_1^\nu \epsilon_\nu^{(\lambda')}(p_2) \tau_i \cr
	& = i g_{DD^* \pi}^2 \epsilon_\mu^{(\lambda)} (p_1) \epsilon_\nu^{(\lambda')} (p_2)
	\frac{k_2^\mu k_1^\nu}{s - M_D^2} \tau_j \tau_i
\end{align}
The total $D$-exchange diagram :
\begin{align}
	i\mathcal{M}_{ij,\lambda \lambda'}^{D,\mathrm{tot}} = i g_{DD^* \pi}^2 \left\{
	\mathcal{M}_{\lambda \lambda'}^{(+)}\delta_{ij} + \mathcal{M}_{\lambda \lambda'}^{(-)}
	\frac{1}{2} [\tau_i, \tau_j] \right\},
\end{align}
where
\begin{align}
	 &\mathcal{M}^{(+)}_{\lambda \lambda'} = g_{D D^* \pi}^2
	\epsilon_\mu^{(\lambda)} (p_1) \epsilon_\nu^{(\lambda')} (p_2) \left[
	\frac{k_1^\mu k_2^\nu}{s - M_D^2} + \frac{k_2^\mu k_1^\nu}{u - M_D^2} \right] \\
	 &\mathcal{M}^{(-)}_{\lambda \lambda'} = g_{D D^* \pi}^2
	\epsilon_\mu^{(\lambda)} (p_1) \epsilon_\nu^{(\lambda')} (p_2) \left[
	\frac{k_1^\mu k_2^\nu}{s - M_D^2} - \frac{k_2^\mu k_1^\nu}{u - M_D^2} \right]
\end{align}

% ===================================================================================================================

\subsection{$D^*$-exchange}
\begin{figure}[H]
	\centering
	\includegraphics[scale=0.6]{DstarDstar_Dstar-crop.pdf}
	\caption{$D^* \bar{D}^* \rightarrow \pi\pi$ Born diagrams ($D^*$-exchange)}
	\label{fig5}
\end{figure}
There is another possible diagrams that interchange $D^*$ meson.
The corresponding Born diagrams are given in Fig.(\ref{fig5}).
A correlation function is given as
\begin{align}
	\langle \pi(k_1) \pi(k_2) \left\{ i g_{D^*D^*\pi} \varepsilon^{\alpha \beta \mu \nu}
	D_\alpha^* \bm{\tau} \cdot \partial_\mu \bm{\pi} \partial_\nu D_\beta^* \right\}
	\left\{ i g_{D^*D^*\pi} \varepsilon^{\gamma \delta \mu' \nu'} D_\gamma^* \bm{\tau}
	\cdot \partial_{\mu'} \bm{\pi} \partial_{\nu'} D_\delta^* \right\} D^* (p_1) D^* (p_2) \rangle,
\end{align}
where $\langle \cdots \rangle$ means the expectation value of the time ordered operator. The external particles, two $D^*$ and
two pions, extracted by LSZ reduction. In order to calculate the amplitudes, we have to consider all possible contractions
corresponding to the Feynman diagram. There are two derivatives for a pion and a $D^*$ field respectively.
In the case of $t$-channel, the term $\partial_\mu \pi$ just gives a factor of $-ik_{1\mu}$.
However $\partial_\beta D^*$ can be contracted with both of the external and intermediate field.
\begin{align}
	i\mathcal{M}_{ij}^{D^*,s} = i g_{D^*D^*\pi}^2 \varepsilon^{\alpha \beta \mu \nu}
	\varepsilon^{\gamma \delta \mu' \nu'} \epsilon_\beta^* \epsilon_\delta^* (p_1 + q)_\nu
	(p_2 - q)_{\nu'} k_{1\mu} k_{2\mu'} \frac{g_{\alpha \gamma} - \frac{q_\alpha q_\gamma}{m_{D^*}^2}}
	{s - m_{D^*}^2} \tau_i \tau_j
\end{align}
By using the four-momentum conservation, we may write
\begin{align}
	i\mathcal{M}_{ij}^{D^*,s} = i g_{D^*D^*\pi}^2 \varepsilon^{\alpha \beta \mu \nu}
	\varepsilon^{\gamma \delta \mu' \nu'} \epsilon_\beta^* \epsilon_\delta^* (2p_1 - k_1)_\nu
	(2p_2 - k_2)_{\nu'} k_{1\mu} k_{2\mu'} \frac{g_{\alpha \gamma} - \frac{q_\alpha q_\gamma}{m_{D^*}^2}}
	{s - m_{D^*}^2} \tau_i \tau_j .
\end{align}
It is easy to show that two momenta contracted with Levi-civita tensor is zero as
\begin{align}
	\varepsilon^{\alpha \beta \mu \nu} k_\mu k_\nu C_{\alpha \beta}
	= - \epsilon^{\alpha \beta \nu \mu} k_\mu k_\nu C_{\alpha \beta}
	= - \epsilon^{\alpha \beta \nu \mu} k_\nu k_\nu C_{\alpha \beta}
	= - \varepsilon^{\alpha \beta \mu \nu} k_\mu k_\nu C_{\alpha \beta} = 0,
\end{align}
where the $C_{\alpha \beta}$ is arbitrary rank 2 tensor. We thus have
\begin{align}
	i\mathcal{M}_{ij}^{D^*,s} = 4i g_{D^*D^*\pi}^2 \varepsilon^{\alpha \beta \mu \nu}
	\varepsilon^{\gamma \delta \mu' \nu'} \epsilon_\beta^* \epsilon_\delta^* p_{1\nu}
	p_{2\nu'} k_{1\mu} k_{2\mu'} \frac{g_{\alpha \gamma} - \frac{q_\alpha q_\gamma}{m_{D^*}^2}}
	{s - m_{D^*}^2} \tau_i \tau_j .
\end{align}
In the same way, we calculate the $u$-channel amplitude.
\begin{align}
	i\mathcal{M}_{ij}^{D^*,u} = 4i g_{D^*D^*\pi}^2 \varepsilon^{\alpha \beta \mu \nu}
	\varepsilon^{\gamma \delta \mu' \nu'} \epsilon_\beta^* \epsilon_\delta^* p_{1\nu}
	p_{2\nu'} k_{2\mu} k_{1\mu'} \frac{g_{\alpha \gamma} - \frac{q_\alpha q_\gamma}{m_{D^*}^2}}
	{u - m_{D^*}^2} \tau_j \tau_i
\end{align}
From the symmetry of momenta $p_i$ or $k_i$, the $q_\alpha q_\gamma$ term vanishes, so that
the $D^*-$exchange Feynman amplitudes can be written as
\begin{align}
	\mathcal{M}_{ij}^{D^*,s} = 4g_{D^*D^*\pi}^2 \varepsilon^{\alpha \beta \mu \nu}
	\varepsilon^{\gamma \delta \mu' \nu'} \epsilon_\beta^* \epsilon_\delta^*
	g_{\alpha \gamma} p_{1\nu} p_{2\nu'} \frac{k_{1\mu} k_{2\mu'}}
	{s - m_{D^*}^2} \tau_i \tau_j
\end{align}
\begin{align}
	\mathcal{M}_{ij}^{D^*,u} = 4g_{D^*D^*\pi}^2 \varepsilon^{\alpha \beta \mu \nu}
	\varepsilon^{\gamma \delta \mu' \nu'} \epsilon_\beta^* \epsilon_\delta^*
	g_{\alpha \gamma} p_{1\nu} p_{2\nu'} \frac{k_{2\mu} k_{1\mu'}}
	{u - m_{D^*}^2} \tau_j \tau_i
\end{align}
The total $D^*$-exchange diagram :
\begin{align}
	\mathcal{M}_{ij}^{D^*} = \mathcal{M}_{D^*}^{(+)} \delta_{ij} + \mathcal{M}_{D^*}^{(-)}
	\frac{1}{2} [\tau_i, \tau_j],
\end{align}
where
\begin{align}
	\mathcal{M}_{D^*}^{(+)} = 4g_{D^*D^*\pi}^2 \varepsilon^{\alpha \beta \mu \nu}
	\varepsilon^{\gamma \delta \mu' \nu'} \epsilon_\beta^* \epsilon_\delta^*
	g_{\alpha \gamma} p_{1\nu} p_{2\nu'} \left[
	\frac{k_{1\mu} k_{2\mu'}}{s - m_{D^*}^2} + \frac{k_{2\mu} k_{1\mu'}}{u - m_{D^*}^2}\right]
\end{align}
\begin{align}
	\mathcal{M}_{D^*}^{(-)} = 4g_{D^*D^*\pi}^2 \varepsilon^{\alpha \beta \mu \nu}
	\varepsilon^{\gamma \delta \mu' \nu'} \epsilon_\beta^* \epsilon_\delta^*
	g_{\alpha \gamma} p_{1\nu} p_{2\nu'} \left[
	\frac{k_{1\mu} k_{2\mu'}}{s - m_{D^*}^2} - \frac{k_{2\mu} k_{1\mu'}}{u - m_{D^*}^2}\right]
\end{align}
If we extract polarization vector from amplitudes, it can be expressed as
\begin{align}
	\mathcal{M}_{D^*}^{(+)\beta \delta} = 4g_{D^*D^*\pi}^2 \varepsilon^{\alpha \beta \mu \nu}
	\varepsilon^{\gamma \delta \mu' \nu'} g_{\alpha \gamma} p_{1\nu} p_{2\nu'} \left[
	\frac{k_{1\mu} k_{2\mu'}}{s - m_{D^*}^2} + \frac{k_{2\mu} k_{1\mu'}}{u - m_{D^*}^2}\right]
\end{align}
\begin{align}
	\mathcal{M}_{D^*}^{(-)\beta \delta} = 4g_{D^*D^*\pi}^2 \varepsilon^{\alpha \beta \mu \nu}
	\varepsilon^{\gamma \delta \mu' \nu'} g_{\alpha \gamma} p_{1\nu} p_{2\nu'} \left[
	\frac{k_{1\mu} k_{2\mu'}}{s - m_{D^*}^2} - \frac{k_{2\mu} k_{1\mu'}}{u - m_{D^*}^2}\right].
\end{align}
To summarize, we may write
\begin{align}
	\mathcal{M}_{(+)}^{\beta \delta} = g_{D D^* \pi}^2 \left[
	\frac{k_1^\beta k_2^\delta}{s - M_D^2} + \frac{k_2^\beta k_1^\delta}{u - M_D^2} \right]
	+ 4g_{D^*D^*\pi}^2 \varepsilon^{\alpha \beta \mu \nu}
	\varepsilon^{\gamma \delta \mu' \nu'} g_{\alpha \gamma} p_{1\nu} p_{2\nu'} \left[
	\frac{k_{1\mu} k_{2\mu'}}{s - m_{D^*}^2} + \frac{k_{2\mu} k_{1\mu'}}{u - m_{D^*}^2}\right]
\end{align}
\begin{align}
	\mathcal{M}_{(-)}^{\beta \delta} = g_{D D^* \pi}^2 \left[
	\frac{k_1^\beta k_2^\delta}{s - M_D^2} - \frac{k_2^\beta k_1^\delta}{u - M_D^2} \right]
	+ 4g_{D^*D^*\pi}^2 \varepsilon^{\alpha \beta \mu \nu}
	\varepsilon^{\gamma \delta \mu' \nu'} g_{\alpha \gamma} p_{1\nu} p_{2\nu'} \left[
	\frac{k_{1\mu} k_{2\mu'}}{s - m_{D^*}^2} - \frac{k_{2\mu} k_{1\mu'}}{u - m_{D^*}^2}\right]
\end{align}
% ===================================================================================================================
% ===================================================================================================================
\section{The $\pi D^* \rightarrow \pi D^*$ amplitude}
In order to check the crossing symmetry,the same procedure, we did at
previous sections, may be done in these scattering processes. We will
calculate $\pi D^* \rightarrow \pi D^*$ amplitudes in this section and
represent amplitudes in terms of $s,t,$ and $u$, so that they are related
to $D^* \bar{D}^* \rightarrow \pi\pi$. The corresponding Feynman diagrams :

%	piD^* diagram will be inserted

%	Vertex Feynman rules will be inserted	(appendix?)



\newpage
\section{rescattering equation}
The Blankenbecler-Sugar equation can decribe $\pi\pi$ interaction.
\begin{align} \label{BbS1}
\mathcal{M}_{D^* \bar{D}^* \rightarrow \pi\pi}^
{\lambda_1 \lambda_2} (p,p';s) =
\mathcal{M}_{D^* \bar{D}^* \rightarrow \pi\pi}^
{\mathrm{Born},\lambda_1 \lambda_2} (p,p';s)
+ \int{d^3q\, \frac{1}{(2\pi)^3} \frac{1}{2\omega_q}
\frac{\mathcal{M}_{D^* \bar{D}^* \rightarrow \pi\pi}^
{\mathrm{Born},\lambda_1 \lambda_2}(p,q;s)
\mathcal{T}_{\pi\pi \rightarrow \pi\pi}(q,p';s)}
{s - 4\omega_q^2 + i \varepsilon}}
\end{align}
By means of partial expansion we may reduce this equation.
\begin{align}
\langle \phi_q \theta_q 0 0 |
\mathcal{M}_{D^* \bar{D}^* \rightarrow \pi\pi}^{\lambda_1 \lambda_2}
| 0 0 \lambda_1 \lambda_2 \rangle &= \sum_{J} \sum_{M,M'}
\langle \phi_q \theta_q 0 0 | J M' 0 0 \rangle
\langle J M' 0 0 | \mathcal{M}_{D^* \bar{D}^* \rightarrow \pi\pi}^{J, \lambda_1 \lambda_2}
| J M \lambda_1 \lambda_2 \rangle \langle J M \lambda_1 \lambda_2 |
0 0 \lambda_1 \lambda_2 \rangle \delta_{MM'} \cr
&=\sum_J \sum_{M} \frac{2J+1}{4\pi} D^{J*}_{M0} (\phi_q, \theta_q, 0)
D^{J}_{M \lambda} (0,0,0)
\mathcal{M}_{D^* \bar{D}^* \rightarrow \pi\pi}^{J, \lambda_1 \lambda_2}
\end{align}

\begin{align}
\langle \phi \theta 0 0 | \mathcal{T}_{\pi\pi \rightarrow \pi\pi}
| \phi_q \theta_q 0 0 \rangle &=
\sum_{J} \sum_{M} \frac{2J+1}{4\pi}
D^{J*}_{M 0} (\phi, \theta, 0) D^J_{M 0} (\phi_q, \theta_q, 0)
\mathcal{T}_{\pi\pi \rightarrow \pi\pi}^J
\end{align}

\begin{align} \label{BbS2}
\sum_J {\frac{2J+1}{4\pi}\mathcal{M}_{D^* \bar{D}^* \rightarrow \pi\pi} ^
{J, \lambda_1 \lambda_2} D^{J*}_{\lambda 0} (\phi, \theta, 0)}
= &\sum_J {\frac{2J+1}{4\pi} \mathcal{M}_{D^* \bar{D}^* \rightarrow \pi\pi}
^{\mathrm{Born}, J ; \lambda_1 \lambda_2}} D^{J*}_{\lambda 0} (\phi, \theta, 0) \cr
&+ \sum_{J,J'} \sum_{M,M'} {\int{d^3q\, \frac{(2J+1)(2J'+1)}{(4\pi)^2}
\frac{1}{(2\pi)^3 2\omega_q (s - 4\omega_q^2 + i \varepsilon)}}}
\cr &\times
\mathcal{M}_{D^* \bar{D}^* \rightarrow \pi\pi}^{\mathrm{Born},J, \lambda_1 \lambda_2}
\mathcal{T}_{\pi\pi \rightarrow \pi\pi}^{J'}
 D^{J}_{M \lambda} (0,0,0) D^{J*}_{M' 0} (\phi, \theta, 0) \cr
&\times D^J_{M 0} (\phi_q, \theta_q, 0) D^{J*}_{M'0} (\phi_q, \theta_q, 0) \cr
= &\sum_J \frac{2J+1}{4\pi} \mathcal{M}_{D^* \bar{D}^* \rightarrow \pi\pi}
^{\mathrm{Born},J;\lambda_1 \lambda_2} D^{J*}_{\lambda 0} (\phi, \theta, 0) \cr
& + \sum_J \int{dq\, q^2 \frac{2J+1}{4\pi}
\frac{1}{(2\pi)^3 2\omega_q (s - 4\omega_q^2 + i \varepsilon)}} \cr
& \times \mathcal{M}_{D^* \bar{D}^* \rightarrow \pi\pi}^{\mathrm{Born},J, \lambda_1 \lambda_2}
\mathcal{T}_{\pi\pi \rightarrow \pi\pi}^{J} D^{J*}_{\lambda 0} (\phi, \theta, 0),
\end{align}
so that for the scalar channel we obtain
\begin{align}
\mathcal{M}_{D^* \bar{D}^* \rightarrow \pi\pi}^{J=0, \lambda_1 \lambda_2}
= \mathcal{M}_{D^* \bar{D}^* \rightarrow \pi\pi}^
{\mathrm{Born},J=0; \lambda_1 \lambda_2} +
\int{dq\, q^2 \frac{1}{(2\pi)^3 2\omega_q (s - 4\omega_q^2 + i \varepsilon)}}
\mathcal{M}_{D^* \bar{D}^* \rightarrow \pi\pi}^{\mathrm{Born},J=0, \lambda_1 \lambda_2}
\mathcal{T}_{\pi\pi \rightarrow \pi\pi}^{J=0}
\end{align}

\section{Form factor dependence}
We have calculated the Born amplitudes so far. Consequently, because of the several momenta in their numerator,
they are badly divergent. Thus we shall introduce the form factors. Several kinds of form factors can be considered.
We first try a monopole type form factor.
\begin{align}
	F = \frac{\Lambda^2 - m_{D^*}^2}{\Lambda^2 + |\mathbf{k}|^2}
\end{align}
\begin{figure}[H]
	\centering
	\includegraphics[scale=0.8]{forms.pdf}
	\caption{The blue, yellow and green lines correspond to monopole, $s$-channel and exponential form factor respectively.
	$\Lambda=2400$ and $\cos{\theta}=0$ for $s$-channel form factor.}
	\label{form}
\end{figure}
As depicted in Fig.(\ref{Born monoff}), this form factor is not sufficient to supress the high energy divergences.
\begin{figure}[H]
	\centering
	\includegraphics[scale=0.5]{Spectral_Born_mono_2000_171031.pdf}
	\caption{The $D \bar{D} \rightarrow \pi\pi$ Born amplitude. A monopole form facor is used from $\Lambda=2000$ to $2400$}
	\label{Born monoff}
\end{figure}
We may also take account of the $s$-channel and exponential form factors.
\begin{align}
	F_{\mathrm{s}} = \frac{\Lambda^4}{\Lambda^4 + \left( s - m_{D^*}^2 \right)}
\end{align}
\begin{align}
	F_{\mathrm{exp}} = \exp{-\frac{|\mathbf{k}|^2}{\Lambda^2}}
\end{align}
However no one was able to supress them.
\begin{figure}[!t]
	\centering
	\includegraphics[scale=0.5]{Spectral_Born_2000_171030.pdf}
	\caption{The $D \bar{D} \rightarrow \pi\pi$ Born amplitude with the $s$-channel form facor from $\Lambda=2000$ to $2400$}
	\label{schff}
\end{figure}
%\begin{figure}[H]
%	\centering
%	\includegraphics[scale=0.5]{Spectral_Born_exp_2000_171031.pdf}
%	\caption{The $D \bar{D} \rightarrow \pi\pi$ Born amplitude with the exponential form facor from $\Lambda=%2000$ to $2400$}
%	\label{expff}
%
%	F = \exp{-\frac{q^2 + m_{D^*}^2}{\Lambda^2}}.
%\end{align}

\begin{figure}[H]
	\centering
	\includegraphics[scale=0.5]{DDstrBorn00_exp_1000.pdf}
	\caption{The $D^* \bar{D^*} \rightarrow \pi\pi$ Born amplitude with the exponential form facor from $\Lambda=1000$ to $1400$}
	\label{expff2}
\end{figure}
\begin{figure}[H]
	\centering
	\includegraphics[scale=0.5]{DDstrBorn11_exp_1000.pdf}
	\caption{The $D^* \bar{D^*} \rightarrow \pi\pi$ Born amplitude with the exponential form facor from $\Lambda=1000$ to $1400$}
	\label{expff3}
\end{figure}


\section{spectral function}
\subsection{Two-body unitarity}
The fact $S$-matrix is unitary implies that $SS^\dagger$ should be
the identity operator, $I$. The $S$-matrix can be decomposed as
\begin{align}
	S = 1 + iT = 1 + i (2\pi)^4 \delta^{(4)} (\sum{P}) \mathcal{M},
\end{align}
where $T$ is transfer matrix and $\sum{P}$ is the total external
four-momentum. From the unitarity, we may write
\begin{align} \label{unitarity}
	S_{fn}S_{in}^\dagger &= \left\{\delta_{fn} + i(2\pi)^4 \delta^{(4)}(P_f - P_n) \mathcal{M}_{fn} \right\}
	\left\{\delta_{ni} - i(2\pi)^4 \delta^{(4)}(P_n - P_i) \mathcal{M}_{in}^\dagger \right\} \cr
	&= \delta_{fi} + i(2\pi)^4 \delta^{(4)} (P_f - P_i) \mathcal{M}_{fi} - i(2\pi)^4 \delta^{(4)} (P_f - P_i) \mathcal{M}_{fi}^\dagger \cr
	&\ \ \ +\sum_n {(2\pi)^8 \delta^{(4)}(P_f - P_n) \delta^{(4)}(P_n - P_i) \mathcal{M}_{fn} \mathcal{M}_{ni}^\dagger} \cr
	&= \delta_{fi}.
\end{align}
We want to describe $D^* \bar{D}^*$ final state, Eq.(\ref{unitarity}) implies that
\begin{align} \label{unitarity3}
	-i\left(\mathcal{M}_{D^* \bar{D}^*} - \mathcal{M}_{D^* \bar{D}^*}^\dagger \right) =
	\sum_n {(2\pi)^4 \delta^{(4)}(P_f - P_n) |\mathcal{M}_{{D^* \bar{D}^*} \rightarrow n}|^2}
\end{align}
If we take a intermediate process as $2\pi$ exchange, Eq.(\ref{unitarity3}) takes the form of
\begin{align}
	2\, \mathrm{Im} \mathcal{M}_{D^* \bar{D}^*} = \frac{1}{2}
	\int{(2\pi)^4 \delta^{(4)}(p_1 + p_2 - q_1 - q_2) |\mathcal{M}_{{D^* \bar{D}^*} \rightarrow \pi\pi}|^2} d \Pi_2,
\end{align}
where $\Pi_2$ is the $2\pi$ phase space and the factor 1/2 comes
from two identical pion propagators. More explicitly,
\begin{align}
	2\, \mathrm{Im} \mathcal{M}_{D^* \bar{D}^*} & = \frac{1}{2}
	\int { \frac{d^3 q_1}{(2\pi)^3}\frac{1}{2\omega_1} \frac{d^3 q_2}{(2\pi)^3} \frac{1}{2\omega_2}
	|\mathcal{M}_{{D^* \bar{D}^*} \rightarrow \pi\pi}|^2} (2\pi)^4 \delta^{(4)}(p_1 + p_2 - q_1 - q_2) \cr
	& = \frac{1}{8} \frac{1}{(2\pi)^2} \int { \frac{d^3 q_1 d^3 q_2}{\omega_1 \omega_2}
	|\mathcal{M}_{{D^* \bar{D}^*} \rightarrow \pi\pi}|^2 \delta^{(4)}(p_1 + p_2 - q_1 - q_2)} \cr
	& = \frac{1}{8} \frac{1}{(2\pi)^2} \int {d^3 q \, \frac{1}{\sqrt{q^2 + m_\pi^2}} \frac{1}{\sqrt{(P+q)^2 + m_\pi^2}}
	|\mathcal{M}_{{D^* \bar{D}^*} \rightarrow \pi\pi}|^2 \delta(E_1 + E_2 - \omega_q - \omega_q')}.
\end{align}
We will choose the center of mass frame. Since the total initial three-momentum $P$ is zero in the center of mass frame,
the amplitude can be written as
\begin{align}
	2\, \mathrm{Im} \mathcal{M}_{D^* \bar{D}^*} & = \frac{1}{8} \frac{1}{(2\pi)^2} \int { \frac{d^3 q}{\omega_q^2}
	|\mathcal{M}_{{D^* \bar{D}^*} \rightarrow \pi\pi}|^2 \delta(2E_p - 2 \sqrt{q^2 + m_\pi^2})} \cr
	& = \frac{1}{8} \frac{1}{(2\pi)^2} \int {q^2dq\, d\Omega\, \frac{1}{\omega_q^2}
	|\mathcal{M}_{{D^* \bar{D}^*} \rightarrow \pi\pi}|^2 \frac{1}{2} \delta(E_p - \sqrt{q^2 + m_\pi^2})}  \cr
	& = \frac{1}{16} \frac{1}{(2\pi)^2} \int {dq\, \frac{q^2}{\omega_q^2} \delta(E_p - \sqrt{q^2 + m_\pi^2})}
	\int {d\Omega \, |\mathcal{M}_{{D^* \bar{D}^*} \rightarrow \pi\pi}|^2}
\end{align}
The non-zero and positive point of the delta function exists at
\begin{align}
	q = \sqrt{E_p^2 - m_\pi^2} = \sqrt{\frac{t}{4} - m_\pi^2} = \frac{1}{2} \sqrt{t - 4m_\pi^2}.
\end{align}
We will use a property of the delta function
\begin{align}
	\delta (g(x)) = \sum{ \frac{\delta(x - \alpha_n)}{|g'(\alpha_n)|}},
\end{align}
then
\begin{align}
	\delta(E_p - \sqrt{q^2 + m_\pi^2}) =\left. \frac{\omega_q}{q}
	\right|_{q = \frac{1}{2} \sqrt{t - 4m_\pi^2}}
	\delta (q - \frac{1}{2} \sqrt{t - 4m_\pi^2}).
\end{align}
Therefore, we may write the imaginary part of the $D^* \bar{D}^*$ scattering amplitude.
\begin{align} \label{eq79}
	2\, \mathrm{Im} \mathcal{M}_{D^* \bar{D}^*} & = \left. \frac{1}{16} \frac{1}{(2\pi)^2}
	\frac{q}{\omega_q} \right|_{q = \frac{1}{2} \sqrt{t - 4m_\pi^2}}
	\int {d\Omega \, |\mathcal{M}_{{D^* \bar{D}^*} \rightarrow \pi\pi}|^2} \cr
	& = \frac{1}{16} \frac{1}{(2\pi)^2} \frac{2}{\sqrt{t}} \frac{\sqrt{t - 4m_\pi^2}}{2}
	\int {d\Omega \, |\mathcal{M}_{{D^* \bar{D}^*} \rightarrow \pi\pi}|^2} \cr
	& = \frac{1}{64 \pi^2} \sqrt{\frac{t - 4m_\pi^2}{t}} \int {d\Omega \,
	|\mathcal{M}_{{D^* \bar{D}^*} \rightarrow \pi\pi}|^2}
\end{align}
%------------------------------------------------------

\subsection{Partial wave expansion}
To obtain the state which has the angular momentum as good quantum number, we will expand amplitude partially.
For the scalar channel, the amplitude has very simple structure, but for vector case this procedure might be
somewhat complicated. The orbital angular momentum are also considered in LSJ system.
Moreover, to construct Lorentz invariant states, the helicity basis or LSJ basis should be choosed for this
calculations.
Since the incoming and outgoing particles are free, we can write
\begin{align}
	|D^* \bar{D}^* \rangle = | p\,\phi'\,\theta'\lambda_1 \lambda_2 \rangle,
\end{align}
where $p$ is magnitude of momentum of $D^*(\bar{D}^*)$ meson and $\phi$ is polar angle and $\theta$ is azimuthal angle.
$\lambda_1$ and $\lambda_2$ are helicity quantum numbers of initial particles. We may write the two-pion states in the same way.
\begin{align}
	|\pi \pi \rangle = | q\,\phi\,\theta,\lambda_3=0, \lambda_4=0 \rangle
\end{align}
States of the definite angular momentum can be constructed from
\begin{align}
	|J M \lambda_1 \lambda_2 \rangle = \sqrt{\frac{2J+1}{4\pi}} \int{d\Omega \, D^{J*}_{M \lambda} (\phi, \theta, 0))}
	|\phi\,\theta \lambda_1 \lambda_2 \rangle,
\end{align}
where $\lambda = \lambda_1 - \lambda_2$. From this formula, we can connect two states.
\begin{align}
	\langle \phi\,\theta \lambda'_1 \lambda'_2 |J M \lambda_1 \lambda_2 \rangle &= \sqrt{\frac{2J+1}{4\pi}}
	\int{d\Omega' \, D^{J*}_{M \lambda} (\phi', \theta', 0)}
	\langle \phi\,\theta \lambda'_1 \lambda'_2 |\phi'\,\theta' \lambda_1 \lambda_2 \rangle \cr
	&= \sqrt{\frac{2J+1}{4\pi}} \int{d\Omega' \, D^{J*}_{M \lambda} (\phi', \theta', 0)
	\delta(\cos{\theta} - \cos{\theta'}) \delta(\phi - \phi')
	\delta_{\lambda_1 \lambda'_1} \delta_{\lambda_2 \lambda'_2}} \cr
	&= \sqrt{\frac{2J+1}{4\pi}} D^{J*}_{M \lambda} (\phi, \theta, 0)
\end{align}
\begin{align}
	\langle JM\lambda_1\lambda_2|\phi\,\theta\lambda_1'\lambda_2'\rangle
	&= \sqrt{\frac{2J+1}{4\pi}} \int d\Omega'\, D_{M\lambda}^J(\Omega')
	\delta^{(2)}(\phi,\theta) \delta_{\lambda_1 \lambda_1'} \delta_{\lambda_2 \lambda_2'} \cr
	&=  \sqrt{\frac{2J+1}{4\pi}} D_{M\lambda}^J(\Omega)
\end{align}
Using this relation, we can expand the partial wave amplitude as
\begin{align}
	\langle \phi_{p'}\,\theta_{p'}00| \mathcal{M} | \phi_q\,\theta_q\lambda_1 \lambda_2 \rangle &= \sum_{JM} \sum_{J'M'}
	\langle \phi_{p'}\,\theta_{p'} 00 | J'M'00 \rangle \langle J'M'00 | \mathcal{M} | JM \lambda_1 \lambda_2 \rangle
	\langle JM \lambda_1 \lambda_2 | \phi_q\,\theta_q, \lambda_1 \lambda_2 \rangle \cr
	&= \sum_{JM} \sum_{J'M'} \sqrt{\frac{2J+1}{4\pi}} \sqrt{\frac{2J'+1}{4\pi}}
	D^{J*}_{M' 0} (\phi_{p'}, \theta_{p'}, 0) D^{J}_{M \lambda} (\phi_q, \theta_q, 0)
	\langle 00 | \mathcal{M}^J | \lambda_1 \lambda_2 \rangle \delta_{JJ'} \delta_{MM'} \cr
	&= \sum_{JM} \frac{2J+1}{4\pi} D^{J*}_{M' 0} (\phi_{p'}, \theta_{p'}, 0) D^{J}_{M \lambda} (\phi_q, \theta_q, 0)
	\langle 00 | \mathcal{M}^J | \lambda_1 \lambda_2 \rangle  \cr
	&= \sum_{JM} \frac{2J+1}{4\pi} D^{J*}_{M' 0} (\phi_{p'}, \theta_{p'}, 0) D^{J}_{M \lambda} (\phi_q, \theta_q, 0)
	\mathcal{M}^{J,\lambda_1 \lambda_2}
\end{align}
Therefore, we obtain
\begin{align}
	\mathcal{M}_{D^* \bar{D}^* \rightarrow \pi \pi}^{\lambda_3 \lambda_4} &=
	\sum_{J'M'} \frac{2J'+1}{4\pi} D^{J'*}_{M' \lambda'} (\phi_{p'}, \theta_{p'}, 0) D^{J'}_{M' 0} (\phi_q, \theta_q, 0)
	\mathcal{M}_{D^* \bar{D}^*\rightarrow \pi\pi}^{J',\lambda_3 \lambda_4}
\end{align}
The azimuthal angle $\phi$ can be set arbitrary. We will choose $\phi=0$ for simplicity.
\begin{align}
	\mathcal{M}_{D^* \bar{D}^* \rightarrow \pi \pi}^{\lambda_3 \lambda_4 \dagger} &=
	\sum_{J'M'} \frac{2J'+1}{4\pi} d_{M' \lambda'}^{J'} (\cos{\theta_{p'}}) d_{M' 0}^{J'}(\cos{\theta_q})
	\mathcal{M}_{D^* \bar{D}^*\rightarrow \pi\pi}^{J',\lambda_3 \lambda_4 \dagger}
\end{align}
where $\bar{\theta} = \theta_{p'} - \theta_{q}$. We expand the amplitudes
\begin{align} \label{dstrpi_partial}
	\mathcal{M}_{D^* \bar{D}^* \rightarrow \pi \pi}^{\lambda_1 \lambda_2} &=
	\sum_{JM} \frac{2J+1}{4\pi} d_{M0}^J(\cos{\theta_q}) d_{M\lambda}^J(\cos{\theta_p})
	\mathcal{M}_{D^* \bar{D}^* \rightarrow \pi\pi}^{J,\lambda_1 \lambda_2}.
\end{align}
In order to project out partial amplitude we multiply by $d_{\mu0}^j$ both sides.
\begin{align}
	2\pi \int_{-1}^{1} d(\cos{\theta_q})
	\,\mathcal{M}_{D^* \bar{D}^* \rightarrow \pi\pi}^{\lambda_1 \lambda_2}
	d_{\mu0}^j(\cos{\theta_q})
	&= \int_{-1}^{1} d\cos{\theta_q}\,\sum_{JM} \frac{2J+1}{2}
	d_{\mu0}^j(\cos{\theta_q}) d_{M0}^J(\cos{\theta_q}) d_{M\lambda}^J(\cos{\theta_p})
	\mathcal{M}_{D^* \bar{D}^* \rightarrow \pi\pi}^{J,\lambda_1 \lambda_2} \cr
	&= \mathcal{M}_{D^* \bar{D}^* \rightarrow \pi\pi}^{j,\lambda_1 \lambda_2}
	d_{\mu\lambda}^j(\cos{\theta_{p'}})
\end{align}
We define partial wave amplitudes according to
\begin{align}
	\mathcal{M}_{D^* \bar{D}^* \rightarrow \pi\pi}^{j,\lambda_1 \lambda_2}
	d_{\mu\lambda}^j(\cos{\theta_{p'}})
	= 2\pi \int_{-1}^{1} d(\cos{\theta_q})
	\,\mathcal{M}_{D^* \bar{D}^* \rightarrow \pi\pi}^{\lambda_1 \lambda_2}
	d_{\mu0}^j(\cos{\theta_q}).
\end{align}
We then may write the RHS of Eq.(\ref{eq79}) more explicitly. If we take into
account unpolarized amplitudes and since we fixed $\phi_q=0$, we get
\begin{align} \label{D*D*barpipi_combining}
	\int d(\cos{\theta_q})\, \left|\mathcal{M}_{D^*\bar{D}^*\rightarrow\pi\pi}^{\mathrm{unpol}}\right|^2
	= &\,\sum_{J,J'}\sum_{M,M'}\sum_{\lambda} \int d(\cos{\theta_q})
	\frac{2J+1}{4\pi} \frac{2J'+1}{4\pi}
	d_{M0}^J(\cos{\theta_q}) d_{M'0}^{J'}(\cos{\theta_q}) \cr
	& \quad \times d_{M\lambda}^J(\cos{\theta_p}) d_{M'\lambda'}^{J'}(\cos{\theta_{p'}})
	\mathcal{M}^{J',\lambda_3 \lambda_4 \dagger} \mathcal{M}^{J,\lambda_1 \lambda_2} \cr
	= &\sum_{J,M,\lambda} \frac{2J+1}{4\pi} \frac{2J+1}{4\pi} \frac{2}{2J+1}
	d_{M\lambda}^J(\cos{\theta_p}) d_{M\lambda'}^{J}(\cos{\theta_{p'}})
	\mathcal{M}^{J,\lambda_3 \lambda_4 \dagger} \mathcal{M}^{J,\lambda_1 \lambda_2} \cr
	= & \frac{1}{2\pi} \sum_{J,\lambda} \frac{2J+1}{4\pi} d_{\lambda' \lambda}^J(\cos{\theta})
	\mathcal{M}^{J,\lambda_3 \lambda_4 \dagger} \mathcal{M}^{J,\lambda_1 \lambda_2},
\end{align}
where $\sum_{\lambda}$ describes sum of $\lambda_i$s.
Now let us try to expand the $D^* \bar{D}^*$ amplitudes.
\begin{align} \label{D*D*bar_expansion}
	\mathcal{M}_{D^* \bar{D}^*}^{\mathrm{unpol}} = \sum_{\lambda}
	\mathcal{M}_{D^* \bar{D}^*}^{\lambda_1 \lambda_2, \lambda_3 \lambda_4} &=
	\sum_{J,\lambda} \frac{2J+1}{4\pi} d_{\lambda' \lambda}^J(\cos{\theta})
	\mathcal{M}_{D^* \bar{D}^*}^{J,\lambda_1 \lambda_2, \lambda_3 \lambda_4}
\end{align}
%=========================================================================================================
\subsection{Spectral function}
Comparing Eq.(\ref{D*D*barpipi_combining}) with the imaginary part of
Eq.(\ref{D*D*bar_expansion}), we derive
\begin{align} \label{Cutkosky}
	\mathrm{Im}\,\mathcal{M}_{D^*\bar{D}^*}^{J,\mathrm{unpol}}(t)
	= \frac{1}{128\pi^2} \sqrt{\frac{t-4m_\pi^2}{t}} \sum_\lambda\times
	\frac{1}{2\pi} \mathcal{M}^{J,\lambda_3 \lambda_4 \dagger}(t)
	\mathcal{M}^{J,\lambda_1 \lambda_2}(t).
\end{align}
Since we are only interested in scalar channel, $J=0$ indicates the fact that
$d^{J=0}$ restrict the helicity quantum numbers of $D^*$ and $\bar{D}^*$ mesons
to $\lambda=\lambda' =0$ which means
\begin{align} \label{helicity_restriction}
	\lambda=\lambda_1-\lambda_2=0\ ; \quad \lambda'=\lambda_3-\lambda_4=0.
\end{align}
If we define the spectral function as the imaginary part of unpolarzed $D^*\bar{D^*}$
amplitude
\begin{align} \label{definition_spectralfunction}
	\rho(t) \equiv \mathrm{Im}\,\mathcal{M}_{D^*\bar{D}^*}^\mathrm{unpol}
	= \mathrm{Im}\sum_{\lambda}
	\mathcal{M}_{D^*\bar{D}^*}^{J,\lambda_1\lambda_2,\lambda_3\lambda_4}(t),
\end{align}
finally we arrive at $D^*D^*$ S-channel amplitude via dispersion relation.
\begin{align} \label{D^*D^*S_amplitude}
	\mathcal{M}_{D^*D^*}^S = \int_{4m_\pi^2}^\infty
	dt'\,\frac{\rho(t')}{t-t'}
\end{align}
%=========================================================================================================
\section{selection rules}
Let us consider the parity and G-parity conservation. We can easily construct
selection rules from these symmetries. First up, In order to do that we have to
know parity eigenvalues of two-particle states both of Helicity and LSJ bases.
\begin{align} \label{parity_twopar_Hel}
	\mathcal{P} |JM \lambda_1 \lambda_2 \rangle &= \mathcal{P}
	\sqrt{\frac{2J+1}{4\pi}} \int d\Omega\, D^{J*}_{M\lambda} (\phi,\theta,0)
	|\phi \theta \lambda_1 \lambda_2 \rangle \cr
	&= \sqrt{\frac{2J+1}{4\pi}} \int \Omega\, D^{J*}_{M\lambda}  (\phi, \theta, 0)
	\eta_1 \eta_2 e^{-i\pi S_1} e^{-i \pi S_2}
	|\phi + \pi, \theta - \pi,-\lambda_1 {-\lambda_2} \rangle \cr
	&= \sqrt{\frac{2J+1}{4\pi}} \int \Omega'\, e^{i\pi (J-S_1 -S_2)} D^{J*}_{M,{-\lambda}} (\Omega')
	\eta_1 \eta_2 |\Omega',-\lambda_1 {-\lambda_2} \rangle \cr
	&= \eta_1 \eta_2 (-1)^{J-S_1-S_2} |\Omega', -\lambda_1 {-\lambda_2} \rangle
\end{align}
and
\begin{align} \label{parity_twopar_LSJ}
	\mathcal{P} |JMLS \rangle &= \mathcal{P} \sum_{m m_S} \sum_{m_1 m_2}
	(LmSm_S|JM)(S_1 m_1 S_2 m_2 | S m)
	\int d\Omega\, Y_m^l (\Omega) |\Omega m_1 m_2 \rangle \cr
	&= \sum_{m m_S} \sum_{m_1 m_2}
	(LmSm_S|JM)(S_1 m_1 S_2 m_2 | S m)
	\int d\Omega\, Y_m^l (\Omega)
	\eta_1 \eta_2 |\Omega' m_1 m_2 \rangle \cr
	&= \sum_{m m_S} \sum_{m_1 m_2}
	(LmSm_S|JM)(S_1 m_1 S_2 m_2 | S m)
	\int d\Omega'\, (-1)^l Y_m^l (\Omega')
	\eta_1 \eta_2 |\Omega' m_1 m_2 \rangle  \cr
	&= \eta_1 \eta_2 (-1)^l |JMLS \rangle
\end{align}
If we assume that transition operators are invariant under parity transformation,
the matrix element of helicity amplitude can be reduced as
\begin{align} \label{parity_sel_Hel}
	\langle \lambda_3 \lambda_4 | \mathcal{M} | \lambda_1 \lambda_2 \rangle
	= \eta \langle -\lambda_3 {-\lambda_4} | \mathcal{M} | -\lambda_1 {-\lambda_2} \rangle\ ;
	\qquad
	\eta \equiv \frac{\eta_1 \eta_2}{\eta_3 \eta_4} (-1)^{S_3+S_4-S_1-S_2}.
\end{align}
From the unitarity condition we may construct selection rules.
Acting the parity and G-parity operation on
$D^* \bar{D}^* \rightarrow \pi\pi$ amplitude, we have the rules as below.
\begin{align} \label{parity_sel_LSJ}
	(-1)^L = \eta_\pi \eta_\pi (-1)^{L_{\pi\pi}} = (-1)^{L_{\pi\pi}}
\end{align}
\begin{align} \label{gparity_sel_LSJ}
	(-1)^{J+I} = G_\pi G_\pi = 1
\end{align}
We are now dealing with $J=0$ and $I=0$ channel.
I listed all possible $\sigma$-channel transition in the Table \ref{tab:template}.
Note that $S$ can be 0, 1, or 2.
\begin{table}[H]
\centering
\begin{tabular}{c c c c c c c}
\hline\hline
 	&	$J^P$	&	$I^G$	&	$L_{\pi\pi}$	&	$S_{\pi\pi}$	&	L	&	S \\
\hline
\multirow{2}*{$\sigma$}	&	\multirow{2}*{$0^+$}	&	\multirow{2}*{$0^+$}	&	0	&	0	&	0	&	0 	\\
		&		&		&	0	&	0	&	2	&	2	\\
\hline\hline
\end{tabular}
\caption{Allowed transitions of $\sigma$-channel} % Table caption, can be commented out if no caption is required
\label{tab:template} % A label for referencing this table elsewhere, references are used in text as \ref{label}
\end{table}

%\section{Results}
%\begin{figure}[H]
%	\centering
%	\includegraphics[scale=0.5]{Spectral_WB_mono_2000_171031.pdf}
%	\caption{The $DD$ sepctral function(Born + Correlated). A monopole form facor is used from $\Lambda=2000$ to $2400$}
%	\label{spectral monoff}
%\end{figure}
%\begin{figure}[H]
%	\centering
%	\includegraphics[scale=0.5]{Spectral_WB_2000_171030.pdf}
%	\caption{The $DD$ sepctral function(Born + Correlated). A $s$-channel form facor is used from $\Lambda=2000$ to $2400$}
%	\label{spectral monoff}
%\end{figure}
%\begin{figure}[H]
%	\centering
%	\includegraphics[scale=0.5]{Spectral_WB_exp_2000_171031.pdf}
%	\caption{The $DD$ sepctral function(Born + Correlated). A exponential form facor is used from $\Lambda=2000$ to $2400$}
%	\label{spectral monoff}
%\end{figure}
\newpage
\appendix
\section{Explicit calculations}
In this section I would like to show the explicit expressions of
LSJ state in terms of the helicity states. It is convenient to
express the LSJ amplitudes that use of a notation :
\begin{align} \label{LSJ_DDpipi}
	\langle L'S' | \mathcal{M}^J | LS \rangle_{\mathrm{LSJ}} \equiv \langle JML'S' | \mathcal{M}^J | JMLS \rangle
\end{align}
From Eq. (\ref{LSJ_to_Hel}) we have
\begin{align} \label{0000LSJ}
	|0000 \rangle_{LSJ} = \sum_{\lambda_1 \lambda_2}
	(000\lambda|0\lambda)(1\lambda_1 1{-\lambda_2}|0\lambda)
	|00\lambda_1 \lambda_2 \rangle
	= \frac{1}{\sqrt{3}}\left(|11\rangle - |00\rangle |-1{-1}\rangle \right)
\end{align}
\begin{align} \label{0022LSJ}
	|0022 \rangle_{LSJ} = \sum_{\lambda_1 \lambda_2}
	(202\lambda|0\lambda)(1\lambda_1 1{-\lambda_2}|2\lambda)
	|00\lambda_1 \lambda_2 \rangle
	=\frac{1}{\sqrt{6}} \left(|11\rangle + 2|00\rangle + |-1{-1}\rangle \right)
\end{align}
Since $D^* \bar{D}^* \rightarrow \pi\pi$ transition only allows
two LSJ amplitudes, four LSJ amplitudes can be taken account.
\begin{align} \label{LSJs}
	\langle 00 | \mathcal{M}^{J=0} | 00 \rangle_{LSJ}, \
	\langle 00 | \mathcal{M}^{J=0} | 22 \rangle_{LSJ}, \
	\langle 22 | \mathcal{M}^{J=0} | 00 \rangle_{LSJ}, \
	\langle 22 | \mathcal{M}^{J=0} | 22 \rangle_{LSJ}.
\end{align}\\
i) Im$\langle00|\mathcal{M}^{J=0}|00\rangle_{LSJ}$
\begin{align} \label{Im0000}
	\mathrm{Im} \langle 00|\mathcal{M}|00\rangle_{LSJ}
	&= \frac{1}{3} \mathrm{Im}
	\Big(\langle 11| - \langle 00| + \langle -1{-1}|\Big) \mathcal{M}
	\Big(|11\rangle - |00\rangle + |-1{-1}\rangle\Big) \cr
	&= \frac{1}{128\pi^2} \sqrt{\frac{t-4m_\pi^2}{t}}
	\frac{1}{3} \cr
	&\quad\ \times \left(
	\mathcal{M}^{00 \dagger}_{D^* \bar{D}^* \rightarrow \pi\pi}
	\mathcal{M}^{00}_{D^* \bar{D}^* \rightarrow \pi\pi}
	- 4 \mathcal{M}^{00 \dagger}_{D^* \bar{D}^* \rightarrow \pi\pi}
	\mathcal{M}^{11}_{D^* \bar{D}^* \rightarrow \pi\pi}
	+ 4 \mathcal{M}^{11 \dagger}_{D^* \bar{D}^* \rightarrow \pi\pi}
	 \mathcal{M}^{11}_{D^* \bar{D}^* \rightarrow \pi\pi}
	\right)
\end{align}\\
ii) Im$\langle00|\mathcal{M}^{J=0}|22\rangle_{LSJ}$
\begin{align} \label{Im0022}
	\mathrm{Im} \langle 00|\mathcal{M}|22\rangle_{LSJ}
	&= \mathrm{Im} \left[ \frac{1}{\sqrt{3}}
	\Big\{ \langle 11| - \langle 00| + \langle -1{-1}|\Big\} \right]
	\mathcal{M} \left[ \frac{1}{\sqrt{6}}
	\Big\{ |11\rangle + 2|00\rangle +|-1{-1}\rangle \Big\} \right] \cr
	&= \frac{1}{128\pi^2} \sqrt{\frac{t-4m_\pi^2}{t}} \frac{1}{3\sqrt{2}}\left(
	-2 \mathcal{M}^{00 \dagger}_{D^* \bar{D}^* \rightarrow \pi\pi}
	\mathcal{M}^{00}_{D^* \bar{D}^* \rightarrow \pi\pi} \right. \cr
	&\qquad\qquad\qquad\left.
	+\, 2 \mathcal{M}^{00 \dagger}_{D^* \bar{D}^* \rightarrow \pi\pi}
	\mathcal{M}^{11}_{D^* \bar{D}^* \rightarrow \pi\pi}
	+4 \mathcal{M}^{11 \dagger}_{D^* \bar{D}^* \rightarrow \pi\pi}
	 \mathcal{M}^{11}_{D^* \bar{D}^* \rightarrow \pi\pi}
	\right)
\end{align}\\
iii) Im$\langle22|\mathcal{M}^{J=0}|22\rangle_{LSJ}$
\begin{align} \label{Im2222}
	\mathrm{Im} \langle 22|\mathcal{M}|22\rangle_{LSJ}
	&= \frac{1}{6} \mathrm{Im}
	\Big\{ \langle 11| +2 \langle 00| + \langle -1{-1}|\Big\}
	\mathcal{M} \Big\{ |11\rangle + 2|00\rangle +|-1{-1}\rangle \Big\} \cr
	&= \frac{1}{128\pi^2} \sqrt{\frac{t-4m_\pi^2}{t}} \frac{2}{3}\left(
	\mathcal{M}^{00 \dagger}_{D^* \bar{D}^* \rightarrow \pi\pi}
	\mathcal{M}^{00}_{D^* \bar{D}^* \rightarrow \pi\pi} \right. \cr
	&\qquad\qquad\qquad\left.
	+\, 2 \mathcal{M}^{00 \dagger}_{D^* \bar{D}^* \rightarrow \pi\pi}
	\mathcal{M}^{11}_{D^* \bar{D}^* \rightarrow \pi\pi}
	+ \mathcal{M}^{11 \dagger}_{D^* \bar{D}^* \rightarrow \pi\pi}
	 \mathcal{M}^{11}_{D^* \bar{D}^* \rightarrow \pi\pi}
	\right)
\end{align}\\

\section{Vertex Feynman rules}
From the effective Lagrangians, Eq.(\ref{Lagrangian}) and Eq.(\ref{Lagrangian_D*}), we derive the vertex Feynman rules. We might read the pion field as
\begin{align*}
	\phi_\pi = \int{\frac{d^3 k}{(2\pi)^3}
	\frac{1}{\sqrt{2 \omega_k}}\left\{a_\mathbf{k}^\dagger e^{ik x} + a_\mathbf{k} e^{-ikx}\right\}}.
\end{align*}
If the pion is created at vertex :
\begin{align*}
	\partial^\mu \phi_\pi | \pi \rangle = &\int{\frac{d^3 k}{(2\pi)^3}
	\frac{1}{\sqrt{2\omega_\mathbf{k}}}\left\{(ik^\mu)a_\mathbf{k}^\dagger e^{ik x} + (-ik^\mu)a_\mathbf{k} e^{-ikx}\right\}} \sqrt{2\omega_q}
	a_\mathbf{q}^\dagger|0\rangle \cr
	=&\int{\frac{d^3 k}{(2\pi)^3}
	\sqrt{\frac{\omega_\mathbf{q}}{\omega_\mathbf{k}}}(-ik^\mu) a_\mathbf{k}  a_\mathbf{q}^\dagger e^{-ikx}}|0\rangle \cr
	=& \int{\frac{d^3 k}{(2\pi)^3}
	\sqrt{\frac{\omega_\mathbf{q}}{\omega_\mathbf{k}}} (-ik^\mu)[a_\mathbf{k},  a_\mathbf{q}^\dagger] e^{-ikx}}|0\rangle \cr
	=& \int{\frac{d^3 k}{(2\pi)^3}
	\sqrt{\frac{\omega_\mathbf{q}}{\omega_\mathbf{k}}}(-ik^\mu) (2\pi)^3 \delta^{(3)} (\vec{k} - \vec{q}) e^{-ikx}}|0\rangle \cr
	=&(-iq^\mu) e^{-iqx} |0\rangle
\end{align*}
If the pion is annihilated at vertex :
\begin{align*}
	\langle \pi | \partial^\mu \phi_\pi = \langle 0| (iq^\mu) e^{iqx}
\end{align*}
This is applied to $D^*D^*\pi$ vertex. The $D^*$ field is given as
\begin{align*}
	D_\alpha^{*(\lambda)} &= \int{\frac{d^3 p}{(2\pi)^3}
	\frac{1}{\sqrt{2 E_p}}\left\{ a_\mathbf{p}^{\lambda \dagger} \epsilon_\alpha^{*(\lambda)}(\mathbf{p})e^{ipx}
	+ b_\mathbf{p}^\lambda \epsilon_\alpha^{(\lambda)}(\mathbf{p}) e^{-ipx}  \right\}},\\
	\partial_\nu D_\alpha^{*(\lambda)} &= \int{\frac{d^3 p}{(2\pi)^3}
	\frac{1}{\sqrt{2 E_p}}\left\{ (ip_\nu) a_\mathbf{p}^{\lambda \dagger} \epsilon_\alpha^{*(\lambda)}(\mathbf{p})e^{ipx}
	+ (-ip_\nu)b_\mathbf{p}^\lambda \epsilon_\alpha^{(\lambda)}(\mathbf{p}) e^{-ipx}  \right\}}.
\end{align*}
If $D^*$ is created at vertex :
\begin{align*}
	D_\alpha^{*(\lambda')} |D^*, \lambda \rangle = \int{\frac{d^3 p'}{(2\pi)^3}
	\frac{1}{\sqrt{2 E_{p'}}} a_\mathbf{p'}^{\lambda'} \epsilon_\alpha^{(\lambda')}(\mathbf{p'}) e^{-ip'x}} \sqrt{2 E_p}
	a_\mathbf{p}^{\lambda \dagger} |0\rangle
	&= \int{\frac{d^3 p'}{(2\pi)^3} \sqrt{\frac{E_{p'}}{E_p}} \epsilon_\alpha^{(\lambda')} (\mathbf{p'})
	a_\mathbf{p'}^{\lambda'} a_\mathbf{p}^{\lambda \dagger} } e^{-ip'x}|0\rangle \cr
	&= \int{\frac{d^3 p'}{(2\pi)^3} \sqrt{\frac{E_{p'}}{E_p}} \epsilon_\alpha^{(\lambda')} (\mathbf{p'})
	a_\mathbf{p'}^{\lambda'} a_\mathbf{p}^{\lambda \dagger} } e^{-ip'x}|0\rangle \cr
	&= \int{\frac{d^3 p'}{(2\pi)^3} \sqrt{\frac{E_{p'}}{E_p}} \epsilon_\alpha^{(\lambda')} (\mathbf{p'})
	(2\pi)^3 \delta^{(3)} (\mathbf{p}-\mathbf{p}') \delta_{\lambda \lambda'}} e^{-ip'x}|0\rangle \cr
	&= \epsilon_\alpha^{(\lambda)}(\mathbf{p}) e^{-ipx} |0\rangle .
\end{align*}
If $D^*$ is annihilated at vertex :
\begin{align*}
	\langle D^*, \lambda | D_\alpha^{*(\lambda')} &= \langle 0 | \sqrt{2E_p} a_\mathbf{p}^\lambda
	\int{\frac{d^3 p'}{(2\pi)^3}} \frac{1}{\sqrt{2E_{p'}}} a_{\mathbf{p}'}^{\lambda' \dagger}
	\epsilon_\alpha^{*(\lambda')} (\mathbf{p}') e^{ip'x} \cr
	&= \langle 0 | \int{\frac{d^3 p'}{(2\pi)^3}} \sqrt{\frac{E_p}{E_{p'}}}
	\epsilon_\alpha^{*(\lambda')} (\mathbf{p}') a_\mathbf{p}^\lambda a_{\mathbf{p}'}^{\lambda' \dagger} e^{ip'x} \cr
	&= \langle 0 | \int{\frac{d^3 p'}{(2\pi)^3}} \sqrt{\frac{E_p}{E_{p'}}}
	\epsilon_\alpha^{*(\lambda')} (\mathbf{p}') (2\pi)^3 \delta^{(3)}(\mathbf{p} - \mathbf{p'}) e^{ip'x} \cr
	&= \langle 0 | \epsilon_\alpha^{*(\lambda)} (\mathbf{p}) e^{ipx}
\end{align*}
Vector propagator :
\begin{align}
	\langle 0 | D_\alpha^* D_\beta^*| 0 \rangle = \frac{-i\left(g_{\alpha \beta} -
	\frac{q_\alpha q_\beta}{m^2}\right)}{q^2 - m^2}
\end{align}
Let us consider the antiparticle fields.

\section{Isospin algebra}
\subsection{Projection operator}
The $D$ and $D^*$ mesons are isodoublet. The isospin pauli matrices are given as
\begin{align}
	\tau_x = \left(\begin{array}{cc}
	0 & 1 \\
	1 & 0
	\end{array} \right), \quad
	\tau_y = \left(\begin{array}{cc}
	0 & i \\
	-i & 0
	\end{array} \right) \quad
	\tau_z = \left(\begin{array}{cc}
	1 & 0 \\
	0 & -1
	\end{array} \right)
\end{align}
We can easily show that they satisfy
%\begin{align}
%	\bm{\tau}^2 = \left(\begin{array}{cc}
%	1 & 0 \\
%	0 & 1
%	\end{array} \right) = 1
%\end{align}
\begin{align}
	\left\{ \tau_i,\tau_j \right\} = 2\delta_{ij},\quad
	\left[ \tau_i, \tau_j \right] = 2 i \varepsilon_{ijk} \tau_k,\quad
	\bm{\tau}^2 = 1.
\end{align}
The eigenvalue of $\bm{\tau}^2$ is
\begin{align}
	\tau^2 |T=1/2,T_z\rangle = 4T(T+1)|T=1/2,T_z\rangle = 3|T=1/2,T_z\rangle
\end{align}
The two isospin projection operator which describe two particles have 0 or 1 total isospin quantum numbers.
If $\bm{\tau}_1$ and $\bm{\tau}_2$ being projection operators of particle 1 and 2, the total isospin operator is
\begin{align}
	\bm{I} = \frac{1}{2} \left( \bm{\tau}_1 + \bm{\tau}_2 \right)
\end{align}
So that
\begin{align}
	\bm{I}^2 = \frac{1}{4} \left(6 + 2\bm{\tau}_1 \cdot \bm{\tau}_2 \right) = \frac{3}{2} + \frac{1}{2}\bm{\tau}_1 \cdot \bm{\tau}_2 .
\end{align}
For $T=0$, the operator $\bm{\tau}_1 \cdot \bm{\tau}_2$ has the following eigenvalue.
\begin{align}
	\bm{\tau}_1 \cdot \bm{\tau}_2 = -3
\end{align}
In case of $T=1$,
\begin{align}
	\bm{\tau}_1 \cdot \bm{\tau}_2 = 1
\end{align}
We thus may construct isospin projection operators as
\begin{align}
	P^0 = \frac{1}{4} - \frac{1}{4}\bm{\tau}_1 \cdot \bm{\tau}_2,\quad
	P_1 = \frac{3}{4} + \frac{1}{4}\bm{\tau}_1 \cdot \bm{\tau}_2
\end{align}
%
\subsection{$\pi\pi$ system}
The pion can be represented in cartesian coorinates as
\begin{align} \label{cartesian pion}
	|\pi_x\rangle = \left( \begin{array}{c c c} 1 \\ 0 \\ 0 \end{array} \right), \quad
	|\pi_y\rangle = \left( \begin{array}{c c c} 0 \\ 1 \\ 0 \end{array} \right), \quad
	|\pi_z\rangle = \left( \begin{array}{c c c} 0 \\ 0 \\ 1 \end{array} \right)
\end{align}
or in spherical coordinates
\begin{align}
	&|\pi^+\rangle = -\frac{1}{\sqrt{2}} (|\pi_x\rangle + i |\pi_y\rangle) = |1,1\rangle\\
	&|\pi^0\rangle = |\pi_z\rangle = |1,0\rangle\\
	&|\pi^-\rangle = \frac{1}{\sqrt{2}} (|\pi_x\rangle - i |\pi_y\rangle) = |1,-1\rangle.
\end{align}
It is easy to show that they are indeed the eigenstates of rotation matrices. As we did for spin operators in Quantum
mechanics, we may construct two pion states which has eigenvalues $T=0$ and $T=1$. For $T=0$
\begin{align}
	|0,0\rangle = \frac{1}{\sqrt{3}} |1,1\rangle |1,-1\rangle - \frac{1}{\sqrt{3}} |1,0\rangle + \frac{1}{\sqrt{3}} |1,-1\rangle.
\end{align}
and for $T=1$
\begin{align}
	|1,1\rangle = \frac{1}{\sqrt{2}} |1,1\rangle |1,0\rangle - \frac{1}{\sqrt{2}} |1,0\rangle |1,1\rangle
\end{align}
\begin{align}
	|1,0\rangle = \frac{1}{\sqrt{2}} |1,1\rangle |1,-1\rangle - \frac{1}{\sqrt{2}} |1,-1\rangle |1,1\rangle
\end{align}
\begin{align}
	|1,-1\rangle = \frac{1}{\sqrt{2}} |1,0\rangle |1,-1\rangle - \frac{1}{\sqrt{2}} |1,-1\rangle |1,0\rangle
\end{align}
They also can be expressed in terms of Eq.(\ref{cartesian pion}).
\begin{align}
	|0,0\rangle_{\pi\pi} = -\frac{1}{\sqrt{3}} \sum_{i=1}^3 |\pi_i\rangle |\pi_i\rangle
	= -\frac{1}{\sqrt{3}} \left(|\pi_1\rangle |\pi_1\rangle + |\pi_2\rangle |\pi_2\rangle
	+|\pi_3\rangle |\pi_3\rangle \right)
\end{align}
\begin{align}
	|1,1\rangle_{\pi\pi} = \frac{1}{2} \left(-|\pi_1\rangle |\pi_3\rangle + |\pi_3\rangle |\pi_1\rangle
	-i|\pi_2\rangle |\pi_3\rangle +i|\pi_3\rangle |\pi_2\rangle  \right)
\end{align}
\begin{align}
	|1,0\rangle_{\pi\pi} = \frac{i}{\sqrt{2}} \left( |\pi_1\rangle |\pi_2\rangle
	-|\pi_2\rangle |\pi_1\rangle\right)
\end{align}
\begin{align}
	|1,-1\rangle_{\pi\pi} = \frac{1}{2} \left(-|\pi_1\rangle |\pi_3\rangle + |\pi_3\rangle |\pi_1\rangle
	+i|\pi_3\rangle |\pi_2 -i|\pi_2\rangle |\pi_3\rangle \right)
\end{align}
(T=2 case)
%
\subsection{$D\bar{D}$ system}
\begin{align}
	|1/2,1/2\rangle_{D} \equiv \chi_{\frac{1}{2}} =
	\left( \begin{array}{c} 1\\0 \end{array} \right),\quad
	|1/2,{-1}/2\rangle_D \equiv \chi_{-\frac{1}{2}} =
	\left( \begin{array}{c} 0\\1 \end{array} \right).
\end{align}
Let us consider $G$-parity operator for consideration of anti-isospinor $\tilde{\chi}_m$.
The $G$ operation can be achieved by rotating $180^{\circ}$ along the $y-$direction in isotopic space and
acting the charge conjugation operator.
\begin{align}
	G = C \exp\left(-i\pi\frac{\tau_y}{2}\right)
\end{align}
We see that
\begin{align}
	\exp\left(-i\pi\frac{\tau_y}{2}\right) = \cos{\frac{\pi}{2}}+i\tau_2\sin{\frac{\pi}{2}} = i\tau_2 =
	\left( \begin{array}{cc} 0&1\\-1&0 \end{array} \right).
\end{align}
\begin{align}
	|1/2,1/2\rangle_{\bar{D}} = \tilde{\chi}_{\frac{1}{2}} = G \chi_{\frac{1}{2}} = \left( \begin{array}{cc} 0&1\\-1&0 \end{array} \right)
	\left( \begin{array}{c} 1\\0 \end{array} \right) = \left( \begin{array}{c} 0\\-1 \end{array} \right) = -\chi_{-\frac{1}{2}}
\end{align}
\begin{align}
	|1/2,-1/2\rangle_{\bar{D}} = \tilde{\chi}_{-\frac{1}{2}} = G \chi_{-\frac{1}{2}} = \left( \begin{array}{cc} 0&1\\-1&0 \end{array} \right)
	\left( \begin{array}{c} 0\\1 \end{array} \right) = \left( \begin{array}{c} 1\\0 \end{array} \right) = \chi_{\frac{1}{2}},
\end{align}
Thus we get
\begin{align}
	\tilde{\chi}_m = (-1)^{m+\frac{1}{2}} \chi_{-m}.
\end{align}
Now we may describe $D\bar{D}$ system for $I=0$ by means of $\chi$ and $\tilde{\chi}$.
\begin{align}
	|0,0\rangle_{D\bar{D}} = \frac{1}{\sqrt{2}} \left(\chi_{\frac{1}{2}} \tilde{\chi}_{-\frac{1}{2}} -
	\chi_{-\frac{1}{2}} \tilde{\chi}_{\frac{1}{2}} \right)
	= \frac{1}{\sqrt{2}} \left(\chi_{\frac{1}{2}}\chi_{\frac{1}{2}} + \chi_{-\frac{1}{2}}\chi_{-\frac{1}{2}} \right).
\end{align}
and for $I=1$
\begin{align}
	|1,1\rangle_{D\bar{D}} = \chi_{\frac{1}{2}} \tilde{\chi}_{\frac{1}{2}} = -\chi_{\frac{1}{2}}\chi_{\frac{1}{2}}
\end{align}
\begin{align}
	|1,0\rangle_{D\bar{D}} = \frac{1}{\sqrt{2}} \left(\chi_{\frac{1}{2}} \tilde{\chi}_{-\frac{1}{2}} +
	\chi_{-\frac{1}{2}} \tilde{\chi}_{\frac{1}{2}} \right)
	= \frac{1}{\sqrt{2}} \left(\chi_{\frac{1}{2}}\chi_{\frac{1}{2}} - \chi_{-\frac{1}{2}}\chi_{-\frac{1}{2}} \right)
\end{align}
\begin{align}
	|1,-1\rangle_{D\bar{D}} = \chi_{-\frac{1}{2}} \tilde{\chi}_{-\frac{1}{2}} = \chi_{-\frac{1}{2}}\chi_{-\frac{1}{2}}
\end{align}
%
\subsection{Projection operator}
The amplitudude operator can be decomposed into two parts :
\begin{align}
	\mathcal{M}_{\alpha \beta} = \mathcal{M}^{(+)}\delta_{\alpha \beta} + \mathcal{M}^{(-)} \frac{1}{2} [\tau_\alpha,\tau_\beta].
\end{align}
Since we are dealing with $D\bar{D}$ and $\pi\pi$ systems as
A matrix element of $\mathcal{M}$ can be obtained by sandwitching
\begin{align}
	\mathcal{M}^{I=0} &= {}_{\pi\pi}\langle 0,0| \mathcal{M} | 0,0 \rangle_{D\bar{D}} \cr
	&= -\frac{1}{\sqrt{3}} \left(\langle\pi_1| \langle\pi_1| + \langle\pi_2| \langle\pi_2|
	+\langle\pi_3| \langle\pi_3| \right)
	\left(\mathcal{M}^{(+)}\delta_{\alpha \beta} + \mathcal{M}^{(-)} \frac{1}{2} [\tau_\alpha,\tau_\beta]\right)
	\frac{1}{\sqrt{2}} \left(\chi_{\frac{1}{2}}\chi_{\frac{1}{2}} + \chi_{-\frac{1}{2}}\chi_{-\frac{1}{2}} \right) \cr
	&= -\frac{1}{\sqrt{6}} \times 3 \left(\chi_{\frac{1}{2}} \mathcal{M}^{(+)} \chi_{\frac{1}{2}}
	+\chi_{-\frac{1}{2}} \mathcal{M}^{(+)} \chi_{-\frac{1}{2}} \right) \cr
	&= -\sqrt{6} \mathcal{M}^{(+)}
\end{align}
\begin{align}
	\mathcal{M}^{I=1} &= {}_{\pi\pi}\langle 1,0|\mathcal{M} | 1,0 \rangle_{D\bar{D}} \cr
	&= \frac{i}{\sqrt{2}} \left( \langle\pi_1| \langle\pi_2| -\langle\pi_2| \langle\pi_1|\right)
	\left(\mathcal{M}^{(+)}\delta_{\alpha \beta} + \mathcal{M}^{(-)} \frac{1}{2} [\tau_\alpha,\tau_\beta]\right)
	\frac{1}{\sqrt{2}} \left(\chi_{\frac{1}{2}}\chi_{\frac{1}{2}} - \chi_{-\frac{1}{2}}\chi_{-\frac{1}{2}} \right) \cr
	&= \frac{i}{4} \mathcal{M}^{(-)} \left(\chi_{\frac{1}{2}}^\dagger \left\{[\tau_1,\tau_2]-[\tau_2,\tau_1]\right\}\chi_{\frac{1}{2}} +
	\chi_{-\frac{1}{2}}^\dagger \left\{[\tau_1,\tau_2]-[\tau_2,\tau_1]\right\} \chi_{-\frac{1}{2}}   \right) \cr
	&= \frac{i}{2} \mathcal{M}^{(-)} \left(\chi_{\frac{1}{2}}^\dagger [\tau_1,\tau_2]\chi_{\frac{1}{2}} +
	\chi_{-\frac{1}{2}}^\dagger [\tau_1,\tau_2] \chi_{-\frac{1}{2}}   \right)\cr
	&= \frac{i}{2} \mathcal{M}^{(-)} \left\{\chi_{\frac{1}{2}}^\dagger (2i\tau_3)\chi_{\frac{1}{2}} +
	\chi_{-\frac{1}{2}}^\dagger (-2i\tau_3) \chi_{-\frac{1}{2}}   \right\} \cr
	&= -2\mathcal{M}^{(-)}.
\end{align}
\begin{align}
	{}_{\pi\pi}\langle 1,1|\mathcal{M} | 1,1 \rangle_{D\bar{D}}
\end{align}
\begin{align}
	{}_{\pi\pi}\langle 1,-1|\mathcal{M} | 1,-1 \rangle_{D\bar{D}}
\end{align}


\end{document}
